%Courses Table
\section{Courses}
	Courses(\underline{courseCode}, \underline{catalogYear}, courseTitle, required) 
		Tuples in the Courses table contain information about a given course during a specific catalog year.

	\subsection{courseCode}
		PK : int 
		Four digit numerical identifier for a class that is given at Texas Tech University. This code is used to represent all classes of this type. There may be many sections that share a course code in a given semester. (e.g. 4000)	
		
	\subsection{catalogYear}
		PK : int(4) 
		Eight-digit numerical identifier for a catalog year, consisting of the concatenation of the two calendar years that make up the academic year. (e.g. We represent the academic year of 2014-2015 as 20142015)
		
	\subsection{courseTitle}	
		: str 
		The description of the class referred to by this tuple.
		
	\subsection{required}
		: boolean
		Whether or not this class is required for a degree in Computer Science (e.g. true = required, false = not required).
	
	
%Sections Table
\section{Sections}
	Sections(\underline{CRN}, \underline{year}, \underline{semester}, sectionNumber, type, days, startTime, endTime, enrollment, capacity, room, bldg) 
		Tuples in the Section table contain information for a unique section of a course.

	\subsection{CRN}
		PK : int  
		Unique ID for the section in a given year.
	
	\subsection{year}
		PK : int (1920 $<$ x $<$ 3000) 
		Year the section is being offered.
		
	\subsection{semester}
		PK : enum (Fall/Spring/Summer I/Summer II)
		Semester the section is being offered.
	
	\subsection{sectionNumber}
	 	: str 
		The unique number of the section.
	
	\subsection{type}
	 	:str 
		Teaching method of the section (e.g. online, in-class and lab).   
	
	
	\subsection{days}
		: str 
		Days the section meets. (e.g. MWF means the class will meet Monday, Wednesday and Friday)
	
	\subsection{startTime}
		: int (0000 $\leq$ x $<$ 2400)
		Time the section starts; the use of military time is for data storage efficiency and accuracy. 12 hour time will be display for the user.
	
	\subsection{endTime}
	 	: int (0000 $\leq$ x $<$ 2400) 
		Time the section ends; the use of military time is for data storage efficiency and accuracy. 12 hour time will be display for the user.
	
	\subsection{enrollment}
		: int  
		Number of students enrolled in the section.
	
	\subsection{capacity}
		: int
		Maximum number of students that can enroll in the section according to the department. 
	\subsection{room}
		: str
		The room number where the class is going to be located.
		    	
	\subsection{bldg}
		: str
		The building where the class is going to be located.
		
	\subsection{description}
		: str
		The title of a special topics or special project course.
		    


%Instructors Table
\section{Instructors}
	Instructors(\underline{rNumber}, lastName, firstName, instructorTitle, tenured, joiningSemester, joiningYear, loadPreference) 
		Tuples in the Instructors table contain information for instructors.
	
	\subsection{rNumber}
		PK: int
		Unique numerical identifier for students, staff, and professors at Texas Tech University.
    
    \subsection{lastName}
		: str  
    	The instructor's last name.
    
    \subsection{firstName}
    	: str 
    	The instructor's first name.
  
    \subsection{instructorTitle}
    	: enum (Professor/FTI/GPTI/Null)
    	Title of the instructor.
    
    \subsection{tenured}    
    	: boolean 
    	Represents if the instructor is tenured or not.
    
    \subsection{joiningSemester}
    	: enum (Fall/Spring/Summer I/Summer II) 
    	The semester in which the instructor joined the college.
    
    \subsection{joiningYear}
    	: int (1920 $<$ x $<$ 3000 )
    	The year in which the instructor joined the college.
    
    \subsection{loadPreference}
    	: enum (Spring/Fall/NULL)
    	The semester in which the instructor prefers to teach more. NULL for no preference.
    	
    
%TAs Table
\section{TAs}
	TAs(\underline{rNumber}, lastName, firstName)
		Tuples in the TAs table represent information about a particular teaching assistant.

	\subsection{rNumber}
		PK : int 
		Unique numerical identifier for students, staff, and professors at Texas Tech University.
		
	\subsection{lastName}
		: str  
    	The TA's last name.
    
    \subsection{firstName}
    	: str 
    	The TA's first name.
	
%Textbooks Table
\section{Textbooks}
	Textbooks(\underline{ISBN}, bookTitle, author, publisher, edition) 
		Tuples in the Textbooks table contain information for books used in a Course.

	\subsection{ISBN}
		PK : big int
		Unique numerical identifier for textbooks.
	
	\subsection{bookTitle}
		: str 
		Title of the textbook.
	
	\subsection{author}
		: str 
		Name of the textbook's author(s).
	
	\subsection{publisher}
		: str 
		Publisher of the textbook.
	
	\subsection{edition}
		: str 
		Edition of the Textbook.


%consistsOf Table
\section{consistsOf}
	consistsOf(\underline{CRN}, \underline{semester}, \underline{year}, courseCode, catalogYear) 
		Tuples in the consistOf table contain information about which course each section is an instance of.
	
	\subsection{CRN}
		PK: int 
		Unique ID for the section in a given year.

	\subsection{semester}
		PK : enum (Fall/Spring/Summer I/Summer II)
		Semester the section is being offered.
	
	\subsection{year}
		PK : int (1920 $<$ x $<$ 3000)
		Year the section is being offered.
	
	\subsection{courseCode}
		: int
		Four digit numerical identifier for a class that is given at Texas Tech University. This code is used to represent all classes of this type. There may be many sections that share a course code in a given semester. (e.g. 4000)
	
	\subsection{catalogYear}
		PK : int(4) 
		Eight-digit numerical identifier for a catalog year, consisting of the concatenation of the two calendar years that make up the academic year. (e.g. We represent the academic year of 2014-2015 as 20142015)


%taughtBy Table
\section{taughtBy}
	taughtBy(\underline{CRN}, \underline{semester}, \underline{year}, \underline{rNumber}) 
		Tuples in the taughtBy table contain information for which instructors teach each section.
 	
 	\subsection{CRN}
		PK: int 
		Unique ID for the section in a given year.
		
	\subsection{semester}
		PK : enum (Fall/Spring/Summer I/Summer II)
		Semester the section is being offered.
	
	\subsection{year}
		PK : int (1920 $<$ x $<$ 3000) 
		Year the section is being offered.
	
	\subsection{rNumber}
		PK : int
		Unique numerical identifier for students, staff, and professors at Texas Tech University.


%HasTA Table
\section{HasTA}
	HasTA(\underline{CRN}, \underline{semester}, \underline{year}, \underline{rNumber}, hoursPerWeek) 
		Tuples in the HasTA table represent a TA being assigned to assist a particular section of a course.
	
	\subsection{CRN}
		PK : int 
		The CRN of the section the TA is assisting.
	
	\subsection{semester}
		PK : enum (Fall/Spring/Summer I/Summer II)
		Semester the section is being offered.
			
	\subsection{year}
		PK : int 
		The year in which the section the TA is assisting is being held.
	
	\subsection{rNumber}
		PK : int 
		Unique numerical identifier for students, staff, and professors at Texas Tech University.
	
	\subsection{hoursPerWeek}
		: real 
		The number of hours each week the TA is available to help (e.g. 16.5).


%usesBook Table
\section{usesBook}
	usesBook(\underline{CRN}, \underline{semester}, \underline{year}, \underline{ISBN}) 
		The usesBook relation gives a description of the textbook that is associated with a given class.
	
	\subsection{CRN}
 		PK : int 
 		Unique ID for the section in a given year.
	
	\subsection{semester}
		PK : enum (Fall/Spring/Summer I/Summer II)
		Semester the section is being offered.
		
	\subsection{year}
		PK : int 
		Year the section is being offered.
	
	\subsection{ISBN}
 		PK : big int  
 		Unique numerical identifier for textbooks.
	
	
%prefers Table	
\section{prefers}
	prefers(\underline{rNumber}, \underline{courseCode}, \underline{catalogYear}, rating) 
		Tuples in the prefers table contain information for relation between a professor and a course preference.
	
	\subsection{rNumber}
		PK : int 
		Unique numerical identifier for students, staff, and professors at Texas Tech University.
	
	\subsection{courseCode}
		PK : int
		Four digit numerical identifier for a class that is given at Texas Tech University. This code is used to represent all classes of this type. There may be many sections that share a course code in a given semester. (e.g. 4000)
	
	\subsection{catalogYear}
		PK : int(4) 
		Eight-digit numerical identifier for a catalog year, consisting of the concatenation of the two calendar years that make up the academic year. (e.g. We represent the academic year of 2014-2015 as 20142015)
	
	\subsection{rating}
		: int 
		Numerical ranking of 1, 2, 3, or null. Ranking represents an instructors desire to teach a course. 1 is the most desirable, followed by 2, then 3. A null value represents that an instructor does not wish to teach the course. 


%requests Table
\section{requests}
	requests(\underline{rNumber}, \underline{courseCode}, \underline{catalogYear}, justification) 
		Tuples in the requests table contain information about which instructors prefer which courses in certain years.
	
	\subsection{rNumber}
		PK : int
		Unique numerical identifier for students, staff, and professors at Texas Tech University.
	
	\subsection{courseCode}
		PK : int
		Four digit numerical identifier for a class that is given at Texas Tech University. This code is used to represent all classes of this type. There may be many sections that share a course code in a given semester. (e.g. 4000)
	
	\subsection{catalogYear}
		PK : int(4) 
		Eight-digit numerical identifier for a catalog year, consisting of the concatenation of the two calendar years that make up the academic year. (e.g. We represent the academic year of 2014-2015 as 20142015)
	
	\subsection{justification}
		: str
		A variable length comment. Justification is a comment written by an instructor to state their reason for wanting (or not) to teach a specific course. 


%Accounts Table
\section{Accounts}
	Accounts(\underline{rNumber}, password, permissionType) 
		Holds account information to control viewing and editing databases on the website.
	
	\subsection{rNumber}
		PK : int 
		Unique numerical identifier for students, staff, and professors at Texas Tech University.
    
    \subsection{password}
    	: char(64)
    	Used to validate the account.
    
    \subsection{permissionType}
	    : enum(Faculty/Instructor/Business) 
		Determines what the user can view and edit.